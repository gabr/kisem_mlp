\section{Implementation}

In this section will be presented simple $C\#$ program as an example implementation of presented topics.
By design it implements the simplest multilayer network with two inputs and one output to model boolean functions.
Program is capable to learn non linearly separable XOR function and show this process.

\subsection{User interface}

User interface presents the network structure.

\begin{figure}[!h]
    \centering
    \includegraphics[scale=0.45]{Media/UI_numbers.png}
    \caption{User interface}
    \label{fig:UI}
\end{figure}

Elements of interface:
\begin{enumerate}[topsep=8pt,itemsep=-1ex,partopsep=1ex,parsep=1ex]
    \item File menu - save and open network configuration with training data
    \item Random weight - set random values for neurons weights
    \item Calculate - calculate network output for given input values (8)
    \item Training data - first two values are network inputs third value is desired network output
    \item Auto train - train network using given Training data
    \item Manual train - perform one iteration of \hyperref[formula:EBP]{EBP algorith} with given input (8) and output given in Result field (7) as training data
    \item Training parameters:
    \begin{enumerate}[topsep=0pt,itemsep=-1ex,partopsep=1ex,parsep=1ex]
        \item Error - desired error value for Auto training (5)
        \item Max steps - limits number of Auto training iterations
        \item Train coef - corresponds to \textit{learn factor} $\alpha$
        \item Result - desired network output value for Manual training
    \end{enumerate}
    \item Network inputs - used for Manual training (6) and for manual network output calculation (3)
    \item Network output - calculated when button Calculate is pressed (3)
\end{enumerate}
\textcolor{ForestGreen}{\textbf{Green}} elements are neurons weights, \textcolor{Cerulean}{\textbf{blue}} are multiplication of weights and inputs, \textcolor{red}{\textbf{red}} are neurons outputs.

\subsection{Program structure}
