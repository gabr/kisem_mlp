\section{Introduction}
Many tasks involving mathematical models and equations cannot be solve by algorithm because of its complexity or limitations in computer calculations. Also such problems exist which are too difficult to be describe by math itself. Sometimes it is even not known if mathematical model exist.

Some of this tasks are trivial for humans and by simulating how the human brain works it is possible to create algorithm that is able to \textit{learn} how to solve a problem. \textit{Learning} mean that mathematical model modify itself in \textit{learning process} to meet certain conditions defined by \textit{training~set}. What is \textit{training set} and how \textit{learning process} is done will be covered in next chapters.

Now it is necessary to point that the way brain works is not fully known. But to create simple model our today knowledge is sufficient enough. Structure and mechanism of single neuron is know so based on that mathematical model can be create. Such model is just a generalization of something very complex. Yet good enough to reach our goals and break the bounders of what computer is capable of.

Algorithms uses neural network approach are relativity easy to understand and write in compare to what they can achieve. They are also quite fast. Because of that they are very popular in the filed of machines learning and problem solving.

Popularity of this method has led to creation of many variants and approaches in neural networks field. From ...